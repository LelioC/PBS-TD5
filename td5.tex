\documentclass[11pt]{article}

    \usepackage[breakable]{tcolorbox}
    \usepackage{parskip} % Stop auto-indenting (to mimic markdown behaviour)
    
    \usepackage{iftex}
    \ifPDFTeX
    	\usepackage[T1]{fontenc}
    	\usepackage{mathpazo}
    \else
    	\usepackage{fontspec}
    \fi

    % Basic figure setup, for now with no caption control since it's done
    % automatically by Pandoc (which extracts ![](path) syntax from Markdown).
    \usepackage{graphicx}
    % Maintain compatibility with old templates. Remove in nbconvert 6.0
    \let\Oldincludegraphics\includegraphics
    % Ensure that by default, figures have no caption (until we provide a
    % proper Figure object with a Caption API and a way to capture that
    % in the conversion process - todo).
    \usepackage{caption}
    \DeclareCaptionFormat{nocaption}{}
    \captionsetup{format=nocaption,aboveskip=0pt,belowskip=0pt}

    \usepackage[Export]{adjustbox} % Used to constrain images to a maximum size
    \adjustboxset{max size={0.9\linewidth}{0.9\paperheight}}
    \usepackage{float}
    \floatplacement{figure}{H} % forces figures to be placed at the correct location
    \usepackage{xcolor} % Allow colors to be defined
    \usepackage{enumerate} % Needed for markdown enumerations to work
    \usepackage{geometry} % Used to adjust the document margins
    \usepackage{amsmath} % Equations
    \usepackage{amssymb} % Equations
    \usepackage{textcomp} % defines textquotesingle
    % Hack from http://tex.stackexchange.com/a/47451/13684:
    \AtBeginDocument{%
        \def\PYZsq{\textquotesingle}% Upright quotes in Pygmentized code
    }
    \usepackage{upquote} % Upright quotes for verbatim code
    \usepackage{eurosym} % defines \euro
    \usepackage[mathletters]{ucs} % Extended unicode (utf-8) support
    \usepackage{fancyvrb} % verbatim replacement that allows latex
    \usepackage{grffile} % extends the file name processing of package graphics 
                         % to support a larger range
    \makeatletter % fix for grffile with XeLaTeX
    \def\Gread@@xetex#1{%
      \IfFileExists{"\Gin@base".bb}%
      {\Gread@eps{\Gin@base.bb}}%
      {\Gread@@xetex@aux#1}%
    }
    \makeatother

    % The hyperref package gives us a pdf with properly built
    % internal navigation ('pdf bookmarks' for the table of contents,
    % internal cross-reference links, web links for URLs, etc.)
    \usepackage{hyperref}
    % The default LaTeX title has an obnoxious amount of whitespace. By default,
    % titling removes some of it. It also provides customization options.
    \usepackage{titling}
    \usepackage{longtable} % longtable support required by pandoc >1.10
    \usepackage{booktabs}  % table support for pandoc > 1.12.2
    \usepackage[inline]{enumitem} % IRkernel/repr support (it uses the enumerate* environment)
    \usepackage[normalem]{ulem} % ulem is needed to support strikethroughs (\sout)
                                % normalem makes italics be italics, not underlines
    \usepackage{mathrsfs}
    

    
    % Colors for the hyperref package
    \definecolor{urlcolor}{rgb}{0,.145,.698}
    \definecolor{linkcolor}{rgb}{.71,0.21,0.01}
    \definecolor{citecolor}{rgb}{.12,.54,.11}

    % ANSI colors
    \definecolor{ansi-black}{HTML}{3E424D}
    \definecolor{ansi-black-intense}{HTML}{282C36}
    \definecolor{ansi-red}{HTML}{E75C58}
    \definecolor{ansi-red-intense}{HTML}{B22B31}
    \definecolor{ansi-green}{HTML}{00A250}
    \definecolor{ansi-green-intense}{HTML}{007427}
    \definecolor{ansi-yellow}{HTML}{DDB62B}
    \definecolor{ansi-yellow-intense}{HTML}{B27D12}
    \definecolor{ansi-blue}{HTML}{208FFB}
    \definecolor{ansi-blue-intense}{HTML}{0065CA}
    \definecolor{ansi-magenta}{HTML}{D160C4}
    \definecolor{ansi-magenta-intense}{HTML}{A03196}
    \definecolor{ansi-cyan}{HTML}{60C6C8}
    \definecolor{ansi-cyan-intense}{HTML}{258F8F}
    \definecolor{ansi-white}{HTML}{C5C1B4}
    \definecolor{ansi-white-intense}{HTML}{A1A6B2}
    \definecolor{ansi-default-inverse-fg}{HTML}{FFFFFF}
    \definecolor{ansi-default-inverse-bg}{HTML}{000000}

    % commands and environments needed by pandoc snippets
    % extracted from the output of `pandoc -s`
    \providecommand{\tightlist}{%
      \setlength{\itemsep}{0pt}\setlength{\parskip}{0pt}}
    \DefineVerbatimEnvironment{Highlighting}{Verbatim}{commandchars=\\\{\}}
    % Add ',fontsize=\small' for more characters per line
    \newenvironment{Shaded}{}{}
    \newcommand{\KeywordTok}[1]{\textcolor[rgb]{0.00,0.44,0.13}{\textbf{{#1}}}}
    \newcommand{\DataTypeTok}[1]{\textcolor[rgb]{0.56,0.13,0.00}{{#1}}}
    \newcommand{\DecValTok}[1]{\textcolor[rgb]{0.25,0.63,0.44}{{#1}}}
    \newcommand{\BaseNTok}[1]{\textcolor[rgb]{0.25,0.63,0.44}{{#1}}}
    \newcommand{\FloatTok}[1]{\textcolor[rgb]{0.25,0.63,0.44}{{#1}}}
    \newcommand{\CharTok}[1]{\textcolor[rgb]{0.25,0.44,0.63}{{#1}}}
    \newcommand{\StringTok}[1]{\textcolor[rgb]{0.25,0.44,0.63}{{#1}}}
    \newcommand{\CommentTok}[1]{\textcolor[rgb]{0.38,0.63,0.69}{\textit{{#1}}}}
    \newcommand{\OtherTok}[1]{\textcolor[rgb]{0.00,0.44,0.13}{{#1}}}
    \newcommand{\AlertTok}[1]{\textcolor[rgb]{1.00,0.00,0.00}{\textbf{{#1}}}}
    \newcommand{\FunctionTok}[1]{\textcolor[rgb]{0.02,0.16,0.49}{{#1}}}
    \newcommand{\RegionMarkerTok}[1]{{#1}}
    \newcommand{\ErrorTok}[1]{\textcolor[rgb]{1.00,0.00,0.00}{\textbf{{#1}}}}
    \newcommand{\NormalTok}[1]{{#1}}
    
    % Additional commands for more recent versions of Pandoc
    \newcommand{\ConstantTok}[1]{\textcolor[rgb]{0.53,0.00,0.00}{{#1}}}
    \newcommand{\SpecialCharTok}[1]{\textcolor[rgb]{0.25,0.44,0.63}{{#1}}}
    \newcommand{\VerbatimStringTok}[1]{\textcolor[rgb]{0.25,0.44,0.63}{{#1}}}
    \newcommand{\SpecialStringTok}[1]{\textcolor[rgb]{0.73,0.40,0.53}{{#1}}}
    \newcommand{\ImportTok}[1]{{#1}}
    \newcommand{\DocumentationTok}[1]{\textcolor[rgb]{0.73,0.13,0.13}{\textit{{#1}}}}
    \newcommand{\AnnotationTok}[1]{\textcolor[rgb]{0.38,0.63,0.69}{\textbf{\textit{{#1}}}}}
    \newcommand{\CommentVarTok}[1]{\textcolor[rgb]{0.38,0.63,0.69}{\textbf{\textit{{#1}}}}}
    \newcommand{\VariableTok}[1]{\textcolor[rgb]{0.10,0.09,0.49}{{#1}}}
    \newcommand{\ControlFlowTok}[1]{\textcolor[rgb]{0.00,0.44,0.13}{\textbf{{#1}}}}
    \newcommand{\OperatorTok}[1]{\textcolor[rgb]{0.40,0.40,0.40}{{#1}}}
    \newcommand{\BuiltInTok}[1]{{#1}}
    \newcommand{\ExtensionTok}[1]{{#1}}
    \newcommand{\PreprocessorTok}[1]{\textcolor[rgb]{0.74,0.48,0.00}{{#1}}}
    \newcommand{\AttributeTok}[1]{\textcolor[rgb]{0.49,0.56,0.16}{{#1}}}
    \newcommand{\InformationTok}[1]{\textcolor[rgb]{0.38,0.63,0.69}{\textbf{\textit{{#1}}}}}
    \newcommand{\WarningTok}[1]{\textcolor[rgb]{0.38,0.63,0.69}{\textbf{\textit{{#1}}}}}
    
    
    % Define a nice break command that doesn't care if a line doesn't already
    % exist.
    \def\br{\hspace*{\fill} \\* }
    % Math Jax compatibility definitions
    \def\gt{>}
    \def\lt{<}
    \let\Oldtex\TeX
    \let\Oldlatex\LaTeX
    \renewcommand{\TeX}{\textrm{\Oldtex}}
    \renewcommand{\LaTeX}{\textrm{\Oldlatex}}
    % Document parameters
    % Document title
    \title{td5}
    
    
    
    
    
% Pygments definitions
\makeatletter
\def\PY@reset{\let\PY@it=\relax \let\PY@bf=\relax%
    \let\PY@ul=\relax \let\PY@tc=\relax%
    \let\PY@bc=\relax \let\PY@ff=\relax}
\def\PY@tok#1{\csname PY@tok@#1\endcsname}
\def\PY@toks#1+{\ifx\relax#1\empty\else%
    \PY@tok{#1}\expandafter\PY@toks\fi}
\def\PY@do#1{\PY@bc{\PY@tc{\PY@ul{%
    \PY@it{\PY@bf{\PY@ff{#1}}}}}}}
\def\PY#1#2{\PY@reset\PY@toks#1+\relax+\PY@do{#2}}

\expandafter\def\csname PY@tok@w\endcsname{\def\PY@tc##1{\textcolor[rgb]{0.73,0.73,0.73}{##1}}}
\expandafter\def\csname PY@tok@c\endcsname{\let\PY@it=\textit\def\PY@tc##1{\textcolor[rgb]{0.25,0.50,0.50}{##1}}}
\expandafter\def\csname PY@tok@cp\endcsname{\def\PY@tc##1{\textcolor[rgb]{0.74,0.48,0.00}{##1}}}
\expandafter\def\csname PY@tok@k\endcsname{\let\PY@bf=\textbf\def\PY@tc##1{\textcolor[rgb]{0.00,0.50,0.00}{##1}}}
\expandafter\def\csname PY@tok@kp\endcsname{\def\PY@tc##1{\textcolor[rgb]{0.00,0.50,0.00}{##1}}}
\expandafter\def\csname PY@tok@kt\endcsname{\def\PY@tc##1{\textcolor[rgb]{0.69,0.00,0.25}{##1}}}
\expandafter\def\csname PY@tok@o\endcsname{\def\PY@tc##1{\textcolor[rgb]{0.40,0.40,0.40}{##1}}}
\expandafter\def\csname PY@tok@ow\endcsname{\let\PY@bf=\textbf\def\PY@tc##1{\textcolor[rgb]{0.67,0.13,1.00}{##1}}}
\expandafter\def\csname PY@tok@nb\endcsname{\def\PY@tc##1{\textcolor[rgb]{0.00,0.50,0.00}{##1}}}
\expandafter\def\csname PY@tok@nf\endcsname{\def\PY@tc##1{\textcolor[rgb]{0.00,0.00,1.00}{##1}}}
\expandafter\def\csname PY@tok@nc\endcsname{\let\PY@bf=\textbf\def\PY@tc##1{\textcolor[rgb]{0.00,0.00,1.00}{##1}}}
\expandafter\def\csname PY@tok@nn\endcsname{\let\PY@bf=\textbf\def\PY@tc##1{\textcolor[rgb]{0.00,0.00,1.00}{##1}}}
\expandafter\def\csname PY@tok@ne\endcsname{\let\PY@bf=\textbf\def\PY@tc##1{\textcolor[rgb]{0.82,0.25,0.23}{##1}}}
\expandafter\def\csname PY@tok@nv\endcsname{\def\PY@tc##1{\textcolor[rgb]{0.10,0.09,0.49}{##1}}}
\expandafter\def\csname PY@tok@no\endcsname{\def\PY@tc##1{\textcolor[rgb]{0.53,0.00,0.00}{##1}}}
\expandafter\def\csname PY@tok@nl\endcsname{\def\PY@tc##1{\textcolor[rgb]{0.63,0.63,0.00}{##1}}}
\expandafter\def\csname PY@tok@ni\endcsname{\let\PY@bf=\textbf\def\PY@tc##1{\textcolor[rgb]{0.60,0.60,0.60}{##1}}}
\expandafter\def\csname PY@tok@na\endcsname{\def\PY@tc##1{\textcolor[rgb]{0.49,0.56,0.16}{##1}}}
\expandafter\def\csname PY@tok@nt\endcsname{\let\PY@bf=\textbf\def\PY@tc##1{\textcolor[rgb]{0.00,0.50,0.00}{##1}}}
\expandafter\def\csname PY@tok@nd\endcsname{\def\PY@tc##1{\textcolor[rgb]{0.67,0.13,1.00}{##1}}}
\expandafter\def\csname PY@tok@s\endcsname{\def\PY@tc##1{\textcolor[rgb]{0.73,0.13,0.13}{##1}}}
\expandafter\def\csname PY@tok@sd\endcsname{\let\PY@it=\textit\def\PY@tc##1{\textcolor[rgb]{0.73,0.13,0.13}{##1}}}
\expandafter\def\csname PY@tok@si\endcsname{\let\PY@bf=\textbf\def\PY@tc##1{\textcolor[rgb]{0.73,0.40,0.53}{##1}}}
\expandafter\def\csname PY@tok@se\endcsname{\let\PY@bf=\textbf\def\PY@tc##1{\textcolor[rgb]{0.73,0.40,0.13}{##1}}}
\expandafter\def\csname PY@tok@sr\endcsname{\def\PY@tc##1{\textcolor[rgb]{0.73,0.40,0.53}{##1}}}
\expandafter\def\csname PY@tok@ss\endcsname{\def\PY@tc##1{\textcolor[rgb]{0.10,0.09,0.49}{##1}}}
\expandafter\def\csname PY@tok@sx\endcsname{\def\PY@tc##1{\textcolor[rgb]{0.00,0.50,0.00}{##1}}}
\expandafter\def\csname PY@tok@m\endcsname{\def\PY@tc##1{\textcolor[rgb]{0.40,0.40,0.40}{##1}}}
\expandafter\def\csname PY@tok@gh\endcsname{\let\PY@bf=\textbf\def\PY@tc##1{\textcolor[rgb]{0.00,0.00,0.50}{##1}}}
\expandafter\def\csname PY@tok@gu\endcsname{\let\PY@bf=\textbf\def\PY@tc##1{\textcolor[rgb]{0.50,0.00,0.50}{##1}}}
\expandafter\def\csname PY@tok@gd\endcsname{\def\PY@tc##1{\textcolor[rgb]{0.63,0.00,0.00}{##1}}}
\expandafter\def\csname PY@tok@gi\endcsname{\def\PY@tc##1{\textcolor[rgb]{0.00,0.63,0.00}{##1}}}
\expandafter\def\csname PY@tok@gr\endcsname{\def\PY@tc##1{\textcolor[rgb]{1.00,0.00,0.00}{##1}}}
\expandafter\def\csname PY@tok@ge\endcsname{\let\PY@it=\textit}
\expandafter\def\csname PY@tok@gs\endcsname{\let\PY@bf=\textbf}
\expandafter\def\csname PY@tok@gp\endcsname{\let\PY@bf=\textbf\def\PY@tc##1{\textcolor[rgb]{0.00,0.00,0.50}{##1}}}
\expandafter\def\csname PY@tok@go\endcsname{\def\PY@tc##1{\textcolor[rgb]{0.53,0.53,0.53}{##1}}}
\expandafter\def\csname PY@tok@gt\endcsname{\def\PY@tc##1{\textcolor[rgb]{0.00,0.27,0.87}{##1}}}
\expandafter\def\csname PY@tok@err\endcsname{\def\PY@bc##1{\setlength{\fboxsep}{0pt}\fcolorbox[rgb]{1.00,0.00,0.00}{1,1,1}{\strut ##1}}}
\expandafter\def\csname PY@tok@kc\endcsname{\let\PY@bf=\textbf\def\PY@tc##1{\textcolor[rgb]{0.00,0.50,0.00}{##1}}}
\expandafter\def\csname PY@tok@kd\endcsname{\let\PY@bf=\textbf\def\PY@tc##1{\textcolor[rgb]{0.00,0.50,0.00}{##1}}}
\expandafter\def\csname PY@tok@kn\endcsname{\let\PY@bf=\textbf\def\PY@tc##1{\textcolor[rgb]{0.00,0.50,0.00}{##1}}}
\expandafter\def\csname PY@tok@kr\endcsname{\let\PY@bf=\textbf\def\PY@tc##1{\textcolor[rgb]{0.00,0.50,0.00}{##1}}}
\expandafter\def\csname PY@tok@bp\endcsname{\def\PY@tc##1{\textcolor[rgb]{0.00,0.50,0.00}{##1}}}
\expandafter\def\csname PY@tok@fm\endcsname{\def\PY@tc##1{\textcolor[rgb]{0.00,0.00,1.00}{##1}}}
\expandafter\def\csname PY@tok@vc\endcsname{\def\PY@tc##1{\textcolor[rgb]{0.10,0.09,0.49}{##1}}}
\expandafter\def\csname PY@tok@vg\endcsname{\def\PY@tc##1{\textcolor[rgb]{0.10,0.09,0.49}{##1}}}
\expandafter\def\csname PY@tok@vi\endcsname{\def\PY@tc##1{\textcolor[rgb]{0.10,0.09,0.49}{##1}}}
\expandafter\def\csname PY@tok@vm\endcsname{\def\PY@tc##1{\textcolor[rgb]{0.10,0.09,0.49}{##1}}}
\expandafter\def\csname PY@tok@sa\endcsname{\def\PY@tc##1{\textcolor[rgb]{0.73,0.13,0.13}{##1}}}
\expandafter\def\csname PY@tok@sb\endcsname{\def\PY@tc##1{\textcolor[rgb]{0.73,0.13,0.13}{##1}}}
\expandafter\def\csname PY@tok@sc\endcsname{\def\PY@tc##1{\textcolor[rgb]{0.73,0.13,0.13}{##1}}}
\expandafter\def\csname PY@tok@dl\endcsname{\def\PY@tc##1{\textcolor[rgb]{0.73,0.13,0.13}{##1}}}
\expandafter\def\csname PY@tok@s2\endcsname{\def\PY@tc##1{\textcolor[rgb]{0.73,0.13,0.13}{##1}}}
\expandafter\def\csname PY@tok@sh\endcsname{\def\PY@tc##1{\textcolor[rgb]{0.73,0.13,0.13}{##1}}}
\expandafter\def\csname PY@tok@s1\endcsname{\def\PY@tc##1{\textcolor[rgb]{0.73,0.13,0.13}{##1}}}
\expandafter\def\csname PY@tok@mb\endcsname{\def\PY@tc##1{\textcolor[rgb]{0.40,0.40,0.40}{##1}}}
\expandafter\def\csname PY@tok@mf\endcsname{\def\PY@tc##1{\textcolor[rgb]{0.40,0.40,0.40}{##1}}}
\expandafter\def\csname PY@tok@mh\endcsname{\def\PY@tc##1{\textcolor[rgb]{0.40,0.40,0.40}{##1}}}
\expandafter\def\csname PY@tok@mi\endcsname{\def\PY@tc##1{\textcolor[rgb]{0.40,0.40,0.40}{##1}}}
\expandafter\def\csname PY@tok@il\endcsname{\def\PY@tc##1{\textcolor[rgb]{0.40,0.40,0.40}{##1}}}
\expandafter\def\csname PY@tok@mo\endcsname{\def\PY@tc##1{\textcolor[rgb]{0.40,0.40,0.40}{##1}}}
\expandafter\def\csname PY@tok@ch\endcsname{\let\PY@it=\textit\def\PY@tc##1{\textcolor[rgb]{0.25,0.50,0.50}{##1}}}
\expandafter\def\csname PY@tok@cm\endcsname{\let\PY@it=\textit\def\PY@tc##1{\textcolor[rgb]{0.25,0.50,0.50}{##1}}}
\expandafter\def\csname PY@tok@cpf\endcsname{\let\PY@it=\textit\def\PY@tc##1{\textcolor[rgb]{0.25,0.50,0.50}{##1}}}
\expandafter\def\csname PY@tok@c1\endcsname{\let\PY@it=\textit\def\PY@tc##1{\textcolor[rgb]{0.25,0.50,0.50}{##1}}}
\expandafter\def\csname PY@tok@cs\endcsname{\let\PY@it=\textit\def\PY@tc##1{\textcolor[rgb]{0.25,0.50,0.50}{##1}}}

\def\PYZbs{\char`\\}
\def\PYZus{\char`\_}
\def\PYZob{\char`\{}
\def\PYZcb{\char`\}}
\def\PYZca{\char`\^}
\def\PYZam{\char`\&}
\def\PYZlt{\char`\<}
\def\PYZgt{\char`\>}
\def\PYZsh{\char`\#}
\def\PYZpc{\char`\%}
\def\PYZdl{\char`\$}
\def\PYZhy{\char`\-}
\def\PYZsq{\char`\'}
\def\PYZdq{\char`\"}
\def\PYZti{\char`\~}
% for compatibility with earlier versions
\def\PYZat{@}
\def\PYZlb{[}
\def\PYZrb{]}
\makeatother


    % For linebreaks inside Verbatim environment from package fancyvrb. 
    \makeatletter
        \newbox\Wrappedcontinuationbox 
        \newbox\Wrappedvisiblespacebox 
        \newcommand*\Wrappedvisiblespace {\textcolor{red}{\textvisiblespace}} 
        \newcommand*\Wrappedcontinuationsymbol {\textcolor{red}{\llap{\tiny$\m@th\hookrightarrow$}}} 
        \newcommand*\Wrappedcontinuationindent {3ex } 
        \newcommand*\Wrappedafterbreak {\kern\Wrappedcontinuationindent\copy\Wrappedcontinuationbox} 
        % Take advantage of the already applied Pygments mark-up to insert 
        % potential linebreaks for TeX processing. 
        %        {, <, #, %, $, ' and ": go to next line. 
        %        _, }, ^, &, >, - and ~: stay at end of broken line. 
        % Use of \textquotesingle for straight quote. 
        \newcommand*\Wrappedbreaksatspecials {% 
            \def\PYGZus{\discretionary{\char`\_}{\Wrappedafterbreak}{\char`\_}}% 
            \def\PYGZob{\discretionary{}{\Wrappedafterbreak\char`\{}{\char`\{}}% 
            \def\PYGZcb{\discretionary{\char`\}}{\Wrappedafterbreak}{\char`\}}}% 
            \def\PYGZca{\discretionary{\char`\^}{\Wrappedafterbreak}{\char`\^}}% 
            \def\PYGZam{\discretionary{\char`\&}{\Wrappedafterbreak}{\char`\&}}% 
            \def\PYGZlt{\discretionary{}{\Wrappedafterbreak\char`\<}{\char`\<}}% 
            \def\PYGZgt{\discretionary{\char`\>}{\Wrappedafterbreak}{\char`\>}}% 
            \def\PYGZsh{\discretionary{}{\Wrappedafterbreak\char`\#}{\char`\#}}% 
            \def\PYGZpc{\discretionary{}{\Wrappedafterbreak\char`\%}{\char`\%}}% 
            \def\PYGZdl{\discretionary{}{\Wrappedafterbreak\char`\$}{\char`\$}}% 
            \def\PYGZhy{\discretionary{\char`\-}{\Wrappedafterbreak}{\char`\-}}% 
            \def\PYGZsq{\discretionary{}{\Wrappedafterbreak\textquotesingle}{\textquotesingle}}% 
            \def\PYGZdq{\discretionary{}{\Wrappedafterbreak\char`\"}{\char`\"}}% 
            \def\PYGZti{\discretionary{\char`\~}{\Wrappedafterbreak}{\char`\~}}% 
        } 
        % Some characters . , ; ? ! / are not pygmentized. 
        % This macro makes them "active" and they will insert potential linebreaks 
        \newcommand*\Wrappedbreaksatpunct {% 
            \lccode`\~`\.\lowercase{\def~}{\discretionary{\hbox{\char`\.}}{\Wrappedafterbreak}{\hbox{\char`\.}}}% 
            \lccode`\~`\,\lowercase{\def~}{\discretionary{\hbox{\char`\,}}{\Wrappedafterbreak}{\hbox{\char`\,}}}% 
            \lccode`\~`\;\lowercase{\def~}{\discretionary{\hbox{\char`\;}}{\Wrappedafterbreak}{\hbox{\char`\;}}}% 
            \lccode`\~`\:\lowercase{\def~}{\discretionary{\hbox{\char`\:}}{\Wrappedafterbreak}{\hbox{\char`\:}}}% 
            \lccode`\~`\?\lowercase{\def~}{\discretionary{\hbox{\char`\?}}{\Wrappedafterbreak}{\hbox{\char`\?}}}% 
            \lccode`\~`\!\lowercase{\def~}{\discretionary{\hbox{\char`\!}}{\Wrappedafterbreak}{\hbox{\char`\!}}}% 
            \lccode`\~`\/\lowercase{\def~}{\discretionary{\hbox{\char`\/}}{\Wrappedafterbreak}{\hbox{\char`\/}}}% 
            \catcode`\.\active
            \catcode`\,\active 
            \catcode`\;\active
            \catcode`\:\active
            \catcode`\?\active
            \catcode`\!\active
            \catcode`\/\active 
            \lccode`\~`\~ 	
        }
    \makeatother

    \let\OriginalVerbatim=\Verbatim
    \makeatletter
    \renewcommand{\Verbatim}[1][1]{%
        %\parskip\z@skip
        \sbox\Wrappedcontinuationbox {\Wrappedcontinuationsymbol}%
        \sbox\Wrappedvisiblespacebox {\FV@SetupFont\Wrappedvisiblespace}%
        \def\FancyVerbFormatLine ##1{\hsize\linewidth
            \vtop{\raggedright\hyphenpenalty\z@\exhyphenpenalty\z@
                \doublehyphendemerits\z@\finalhyphendemerits\z@
                \strut ##1\strut}%
        }%
        % If the linebreak is at a space, the latter will be displayed as visible
        % space at end of first line, and a continuation symbol starts next line.
        % Stretch/shrink are however usually zero for typewriter font.
        \def\FV@Space {%
            \nobreak\hskip\z@ plus\fontdimen3\font minus\fontdimen4\font
            \discretionary{\copy\Wrappedvisiblespacebox}{\Wrappedafterbreak}
            {\kern\fontdimen2\font}%
        }%
        
        % Allow breaks at special characters using \PYG... macros.
        \Wrappedbreaksatspecials
        % Breaks at punctuation characters . , ; ? ! and / need catcode=\active 	
        \OriginalVerbatim[#1,codes*=\Wrappedbreaksatpunct]%
    }
    \makeatother

    % Exact colors from NB
    \definecolor{incolor}{HTML}{303F9F}
    \definecolor{outcolor}{HTML}{D84315}
    \definecolor{cellborder}{HTML}{CFCFCF}
    \definecolor{cellbackground}{HTML}{F7F7F7}
    
    % prompt
    \makeatletter
    \newcommand{\boxspacing}{\kern\kvtcb@left@rule\kern\kvtcb@boxsep}
    \makeatother
    \newcommand{\prompt}[4]{
        \ttfamily\llap{{\color{#2}[#3]:\hspace{3pt}#4}}\vspace{-\baselineskip}
    }
    

    
    % Prevent overflowing lines due to hard-to-break entities
    \sloppy 
    % Setup hyperref package
    \hypersetup{
      breaklinks=true,  % so long urls are correctly broken across lines
      colorlinks=true,
      urlcolor=urlcolor,
      linkcolor=linkcolor,
      citecolor=citecolor,
      }
    % Slightly bigger margins than the latex defaults
    
    \geometry{verbose,tmargin=1in,bmargin=1in,lmargin=1in,rmargin=1in}
    
    

\begin{document}
    
    \maketitle
    
    

    
    \hypertarget{pbs---td5}{%
\section{PBS - TD5}\label{pbs---td5}}

\hypertarget{ruxe9ception-dun-signal-bpsk-canal-gaussien-et-probabilituxe9-derreur}{%
\subsection{Réception d'un signal BPSK: canal gaussien et probabilité
d'erreur}\label{ruxe9ception-dun-signal-bpsk-canal-gaussien-et-probabilituxe9-derreur}}

    \hypertarget{objectifs}{%
\subsection{Objectifs}\label{objectifs}}

\begin{itemize}
\tightlist
\item
  Maitriser la relation fondamentale \(p(t) = \frac{dF}{dt}(t)\).
\item
  Connaître et savoir manipuler la loi normale.
\item
  Appliquer les probabilités à un problème de communication basique.
\item
  Comprendre le lien entre taux d'erreur binaire et SNR.
\end{itemize}

    \hypertarget{ajout-dune-constante}{%
\section{1. Ajout d'une constante}\label{ajout-dune-constante}}

Soit \(X\) une variable aléatoire \emph{quelconque}. On définit la
variable aléatoire \(Y\) de la façon suivante: \[Y = X+2.\]

    

    \hypertarget{b-supposons-que-xsimmathcaln0-sigma_x20.25.}{%
\subsubsection{\texorpdfstring{(b) Supposons que
\(X\sim\mathcal{N}(0, \sigma_{X}^2=0.25)\).}{(b) Supposons que X\textbackslash sim\textbackslash mathcal\{N\}(0, \textbackslash sigma\_\{X\}\^{}2=0.25).}}\label{b-supposons-que-xsimmathcaln0-sigma_x20.25.}}

\hypertarget{i-quelle-est-lexpression-de-p_y}{%
\paragraph{\texorpdfstring{(i) Quelle est l'expression de \(p_Y\)
?}{(i) Quelle est l'expression de p\_Y ?}}\label{i-quelle-est-lexpression-de-p_y}}

\hypertarget{ii-donner-lallure-des-deux-densituxe9s-identifier-leur-maximum-et-les-valeurs-uxe0-123sigma-de-la-moyenne}{%
\subsection{\texorpdfstring{\#\#\#\# (ii) Donner l'allure des deux
densités, identifier leur maximum et les valeurs à \(\{1,2,3\}\sigma\)
de la
moyenne}{\#\#\#\# (ii) Donner l'allure des deux densités, identifier leur maximum et les valeurs à \textbackslash\{1,2,3\textbackslash\}\textbackslash sigma de la moyenne}}\label{ii-donner-lallure-des-deux-densituxe9s-identifier-leur-maximum-et-les-valeurs-uxe0-123sigma-de-la-moyenne}}

    \hypertarget{a-densituxe9-de-probabilituxe9}{%
\subsubsection{(a) Densité de
probabilité}\label{a-densituxe9-de-probabilituxe9}}

Exprimer la densité de probabilité \(p_Y\) de \(Y\) en fonction de
\(p_X\).

\textbf{Rappel:} une densité est reliée à sa fonction de répartition par
\[p(t) = \frac{dF}{dt}(t)\]

    \hypertarget{a-densituxe9-de-probabilituxe9-solution}{%
\subsubsection{(a) Densité de probabilité
(solution)}\label{a-densituxe9-de-probabilituxe9-solution}}

\begin{align}
    p_Y(t)  = \frac{dF_Y(t)}{dt}
            &= \frac{d\mathbb{P}(Y < t)}{dt}\\
            &= \frac{d\mathbb{P}(X+2 < t)}{dt}\\
            &= \frac{d\mathbb{P}(X < t-2)}{dt}\\
            &= \frac{dF_X(t-2)}{dt}\\
            &= 1\times F_X'(t-2)\quad\text{on utilise ici }(u \circ v)' = v' \times (u' \circ v)\\
            &= 1\times p_X(t-2)\\
    p_Y(t)  &= p_X(t-2)
\end{align}

La densité de \(Y\) est simplement un décalage de \(2\) vers la droite
de la densité de \(X\).

    \hypertarget{b-loi-normale}{%
\subsubsection{(b) Loi normale}\label{b-loi-normale}}

Supposons que \(X\sim\mathcal{N}(0, \sigma_X^2=0.25)\). On a donc
\[p_X(t) = \frac{1}{\sqrt{2\pi\sigma_X^2}} \exp\left(-\frac{t^2}{2\sigma_X^2}\right).\]

\hypertarget{i-quelle-est-lexpression-de-p_y}{%
\paragraph{\texorpdfstring{(i) Quelle est l'expression de \(p_Y\)
?}{(i) Quelle est l'expression de p\_Y ?}}\label{i-quelle-est-lexpression-de-p_y}}

    \[p_Y(t) = \frac{1}{\sqrt{2\pi\sigma_X^2}} \exp\left(-\frac{(t-2)^2}{2\sigma_X^2}\right).\]

    \hypertarget{ii-donner-lallure-des-deux-densituxe9s-identifier-leur-maximum-et-les-valeurs-uxe0-123sigma-de-la-moyenne.}{%
\paragraph{\texorpdfstring{(ii) Donner l'allure des deux densités,
identifier leur maximum et les valeurs à \(\{1,2,3\}\sigma\) de la
moyenne.}{(ii) Donner l'allure des deux densités, identifier leur maximum et les valeurs à \textbackslash\{1,2,3\textbackslash\}\textbackslash sigma de la moyenne.}}\label{ii-donner-lallure-des-deux-densituxe9s-identifier-leur-maximum-et-les-valeurs-uxe0-123sigma-de-la-moyenne.}}

    \begin{tcolorbox}[breakable, size=fbox, boxrule=1pt, pad at break*=1mm,colback=cellbackground, colframe=cellborder]
\prompt{In}{incolor}{1}{\boxspacing}
\begin{Verbatim}[commandchars=\\\{\}]
\PY{k+kn}{import} \PY{n+nn}{numpy} \PY{k}{as} \PY{n+nn}{np}
\PY{k+kn}{import} \PY{n+nn}{matplotlib}\PY{n+nn}{.}\PY{n+nn}{pyplot} \PY{k}{as} \PY{n+nn}{plt}

\PY{c+c1}{\PYZsh{} Paramètre des figures}
\PY{n}{plt}\PY{o}{.}\PY{n}{rcParams}\PY{o}{.}\PY{n}{update}\PY{p}{(}
    \PY{p}{\PYZob{}}
        \PY{l+s+s1}{\PYZsq{}}\PY{l+s+s1}{font.size}\PY{l+s+s1}{\PYZsq{}}\PY{p}{:} \PY{l+m+mi}{22}\PY{p}{,}
        \PY{l+s+s1}{\PYZsq{}}\PY{l+s+s1}{grid.linestyle}\PY{l+s+s1}{\PYZsq{}}\PY{p}{:} \PY{l+s+s1}{\PYZsq{}}\PY{l+s+s1}{:}\PY{l+s+s1}{\PYZsq{}}\PY{p}{,}
        \PY{l+s+s1}{\PYZsq{}}\PY{l+s+s1}{grid.color}\PY{l+s+s1}{\PYZsq{}}\PY{p}{:} \PY{l+s+s1}{\PYZsq{}}\PY{l+s+s1}{k}\PY{l+s+s1}{\PYZsq{}}\PY{p}{,}
        \PY{l+s+s1}{\PYZsq{}}\PY{l+s+s1}{lines.linewidth}\PY{l+s+s1}{\PYZsq{}}\PY{p}{:} \PY{l+m+mi}{5}\PY{p}{,}
    \PY{p}{\PYZcb{}}
\PY{p}{)}

\PY{c+c1}{\PYZsh{} Pdf de la loi normale}
\PY{k}{def} \PY{n+nf}{normpdf}\PY{p}{(}\PY{n}{x}\PY{p}{,} \PY{n}{mu}\PY{o}{=}\PY{l+m+mi}{0}\PY{p}{,} \PY{n}{sigma}\PY{o}{=}\PY{l+m+mi}{1}\PY{p}{)}\PY{p}{:}
    \PY{l+s+s2}{\PYZdq{}}\PY{l+s+s2}{Densité de probabilité d}\PY{l+s+s2}{\PYZsq{}}\PY{l+s+s2}{une loi normale de paramètre (mu,sigma).}\PY{l+s+s2}{\PYZdq{}}
    \PY{k}{return} \PY{l+m+mi}{1}\PY{o}{/}\PY{n}{np}\PY{o}{.}\PY{n}{sqrt}\PY{p}{(}\PY{l+m+mi}{2}\PY{o}{*}\PY{n}{np}\PY{o}{.}\PY{n}{pi}\PY{o}{*}\PY{n}{sigma}\PY{o}{*}\PY{o}{*}\PY{l+m+mi}{2}\PY{p}{)} \PY{o}{*} \PY{n}{np}\PY{o}{.}\PY{n}{exp}\PY{p}{(}\PY{o}{\PYZhy{}}\PY{p}{(}\PY{p}{(}\PY{n}{x} \PY{o}{\PYZhy{}} \PY{n}{mu}\PY{p}{)}\PY{o}{/}\PY{p}{(}\PY{n}{np}\PY{o}{.}\PY{n}{sqrt}\PY{p}{(}\PY{l+m+mi}{2}\PY{p}{)}\PY{o}{*}\PY{n}{sigma}\PY{p}{)}\PY{p}{)}\PY{o}{*}\PY{o}{*}\PY{l+m+mi}{2}\PY{p}{)}

\PY{c+c1}{\PYZsh{} Paramètre des lois}
\PY{n}{mu\PYZus{}X}\PY{p}{,} \PY{n}{mu\PYZus{}Y} \PY{o}{=} \PY{l+m+mi}{0}\PY{p}{,} \PY{l+m+mi}{2}
\PY{n}{sigma\PYZus{}X}\PY{p}{,} \PY{n}{sigma\PYZus{}Y} \PY{o}{=} \PY{n}{np}\PY{o}{.}\PY{n}{sqrt}\PY{p}{(}\PY{o}{.}\PY{l+m+mi}{25}\PY{p}{)}\PY{p}{,} \PY{n}{np}\PY{o}{.}\PY{n}{sqrt}\PY{p}{(}\PY{o}{.}\PY{l+m+mi}{25}\PY{p}{)}

\PY{c+c1}{\PYZsh{} Abscisse}
\PY{n}{t} \PY{o}{=} \PY{n}{np}\PY{o}{.}\PY{n}{linspace}\PY{p}{(}\PY{o}{\PYZhy{}}\PY{l+m+mi}{2}\PY{p}{,}\PY{l+m+mi}{5}\PY{p}{,}\PY{l+m+mi}{1000}\PY{p}{)}

\PY{c+c1}{\PYZsh{} Densités}
\PY{n}{pdf\PYZus{}X} \PY{o}{=} \PY{n}{normpdf}\PY{p}{(}\PY{n}{t}\PY{p}{,} \PY{n}{mu\PYZus{}X}\PY{p}{,} \PY{n}{sigma\PYZus{}X}\PY{p}{)} \PY{c+c1}{\PYZsh{} densité de X}
\PY{n}{pdf\PYZus{}Y} \PY{o}{=} \PY{n}{normpdf}\PY{p}{(}\PY{n}{t}\PY{p}{,} \PY{n}{mu\PYZus{}Y}\PY{p}{,} \PY{n}{sigma\PYZus{}Y}\PY{p}{)} \PY{c+c1}{\PYZsh{} densité de Y}
\PY{n}{pdf\PYZus{}Y\PYZus{}bis} \PY{o}{=} \PY{n}{normpdf}\PY{p}{(}\PY{n}{t} \PY{o}{\PYZhy{}} \PY{l+m+mi}{2}\PY{p}{,} \PY{n}{mu\PYZus{}X}\PY{p}{,} \PY{n}{sigma\PYZus{}X}\PY{p}{)} \PY{c+c1}{\PYZsh{} densité de Y (obtenue par translation de celle de X)}
\end{Verbatim}
\end{tcolorbox}

    \begin{tcolorbox}[breakable, size=fbox, boxrule=1pt, pad at break*=1mm,colback=cellbackground, colframe=cellborder]
\prompt{In}{incolor}{2}{\boxspacing}
\begin{Verbatim}[commandchars=\\\{\}]
\PY{c+c1}{\PYZsh{} Tracés des courbes}
\PY{n}{fig} \PY{o}{=} \PY{n}{plt}\PY{o}{.}\PY{n}{figure}\PY{p}{(}\PY{n}{figsize}\PY{o}{=}\PY{p}{(}\PY{l+m+mi}{16}\PY{p}{,}\PY{l+m+mi}{4}\PY{p}{)}\PY{p}{,} \PY{n}{dpi}\PY{o}{=}\PY{l+m+mi}{100}\PY{p}{)}\PY{p}{;}
\PY{n}{ax} \PY{o}{=} \PY{n}{plt}\PY{o}{.}\PY{n}{axes}\PY{p}{(}\PY{p}{)}\PY{p}{;}
\PY{n}{ax}\PY{o}{.}\PY{n}{grid}\PY{p}{(}\PY{p}{)}\PY{p}{;}
\PY{n}{ax}\PY{o}{.}\PY{n}{set\PYZus{}xlabel}\PY{p}{(}\PY{l+s+s2}{\PYZdq{}}\PY{l+s+s2}{\PYZdl{}t\PYZdl{}}\PY{l+s+s2}{\PYZdq{}}\PY{p}{)}\PY{p}{;}
\PY{n}{ax}\PY{o}{.}\PY{n}{set\PYZus{}ylabel}\PY{p}{(}\PY{l+s+s2}{\PYZdq{}}\PY{l+s+s2}{\PYZdl{}p(t)\PYZdl{}}\PY{l+s+s2}{\PYZdq{}}\PY{p}{)}\PY{p}{;}
\PY{n}{ax}\PY{o}{.}\PY{n}{plot}\PY{p}{(}\PY{n}{t}\PY{p}{,} \PY{n}{pdf\PYZus{}X}\PY{p}{,} \PY{n}{label}\PY{o}{=}\PY{l+s+s2}{\PYZdq{}}\PY{l+s+s2}{\PYZdl{}p\PYZus{}X(t)\PYZdl{}}\PY{l+s+s2}{\PYZdq{}}\PY{p}{,} \PY{n}{color}\PY{o}{=}\PY{p}{(}\PY{o}{.}\PY{l+m+mi}{25}\PY{p}{,} \PY{l+m+mf}{0.65}\PY{p}{,} \PY{l+m+mf}{0.8}\PY{p}{)}\PY{p}{)}
\PY{n}{ax}\PY{o}{.}\PY{n}{plot}\PY{p}{(}\PY{n}{t}\PY{p}{,} \PY{n}{pdf\PYZus{}Y}\PY{p}{,} \PY{n}{label}\PY{o}{=}\PY{l+s+s2}{\PYZdq{}}\PY{l+s+s2}{\PYZdl{}p\PYZus{}Y(t)\PYZdl{}}\PY{l+s+s2}{\PYZdq{}}\PY{p}{,} \PY{n}{color}\PY{o}{=}\PY{p}{(}\PY{l+m+mi}{1}\PY{p}{,} \PY{l+m+mf}{0.8}\PY{p}{,} \PY{l+m+mf}{0.5}\PY{p}{)}\PY{p}{)}
\PY{n}{ax}\PY{o}{.}\PY{n}{plot}\PY{p}{(}\PY{n}{t}\PY{p}{,} \PY{n}{pdf\PYZus{}Y\PYZus{}bis}\PY{p}{,} \PY{l+s+s2}{\PYZdq{}}\PY{l+s+s2}{\PYZhy{}\PYZhy{}}\PY{l+s+s2}{\PYZdq{}}\PY{p}{,} \PY{n}{label}\PY{o}{=}\PY{l+s+s2}{\PYZdq{}}\PY{l+s+s2}{\PYZdl{}p\PYZus{}X(t\PYZhy{}2)\PYZdl{}}\PY{l+s+s2}{\PYZdq{}}\PY{p}{,} \PY{n}{color}\PY{o}{=}\PY{p}{(}\PY{l+m+mf}{0.25}\PY{p}{,} \PY{l+m+mf}{0.4}\PY{p}{,} \PY{o}{.}\PY{l+m+mi}{8}\PY{p}{)}\PY{p}{,} \PY{n}{dashes}\PY{o}{=}\PY{p}{(}\PY{l+m+mi}{5}\PY{p}{,} \PY{l+m+mi}{5}\PY{p}{)}\PY{p}{)}
\PY{n}{ax}\PY{o}{.}\PY{n}{legend}\PY{p}{(}\PY{p}{)}\PY{p}{;}
\end{Verbatim}
\end{tcolorbox}

    \begin{center}
    \adjustimage{max size={0.9\linewidth}{0.9\paperheight}}{output_11_0.png}
    \end{center}
    { \hspace*{\fill} \\}
    
    \hypertarget{fonctions-mathrmerfcx-et-mathrmqx}{%
\section{\texorpdfstring{2. Fonctions \(\mathrm{erfc}(x)\) et
\(\mathrm{Q}(x)\)}{2. Fonctions \textbackslash mathrm\{erfc\}(x) et \textbackslash mathrm\{Q\}(x)}}\label{fonctions-mathrmerfcx-et-mathrmqx}}

Soit \(U \sim \mathcal{N}(0,1)\). La densité de \(U\) est donnée par
\[p_{U}(t)=\frac{1}{\sqrt{2 \pi}} \exp \left(-\frac{t^{2}}{2}\right).\]

\hypertarget{b-idem-avec-x-simmathcaln0sigma_x2.}{%
\subsection{\texorpdfstring{\#\#\# (b) Idem avec
\(X \sim\mathcal{N}(0,\sigma_{X}^2)\).}{\#\#\# (b) Idem avec X \textbackslash sim\textbackslash mathcal\{N\}(0,\textbackslash sigma\_\{X\}\^{}2).}}\label{b-idem-avec-x-simmathcaln0sigma_x2.}}

    \hypertarget{a-probabilituxe9-derreur-mathbbpu-alpha}{%
\subsubsection{\texorpdfstring{(a) Probabilité d'erreur
\(\mathbb{P}(U > \alpha)\)}{(a) Probabilité d'erreur \textbackslash mathbb\{P\}(U \textgreater{} \textbackslash alpha)}}\label{a-probabilituxe9-derreur-mathbbpu-alpha}}

Les fonctions \(\mathrm{Q}\) et \(\mathrm{erfc}\) sont les suivantes:
\begin{align}
    \operatorname{Q}(x) = \frac{1}{\sqrt{2 \pi}} \int_{x}^{+\infty} e^{\frac{-t^{2}}{2}} \mathrm{d} t \qquad\text{et}\qquad
    \operatorname{erfc}(x)=\frac{2}{\sqrt{\pi}} \int_{x}^{+\infty} e^{-t^{2}} \mathrm{d} t
\end{align}

Commencer par écrire \begin{align}
\mathbb{P}(U > \alpha) = \int_{\alpha}^{+\infty}{p_U(t) \mathrm{d}t} = \int_{\alpha}^{+\infty}{ \frac{1}{\sqrt{2 \pi}} \exp \left(-\frac{t^{2}}{2}\right)\mathrm{d}t}
\end{align}

    \hypertarget{a-probabilituxe9-derreur-mathbbpu-alpha-solution}{%
\subsubsection{\texorpdfstring{(a) Probabilité d'erreur
\(\mathbb{P}(U > \alpha)\)
(solution)}{(a) Probabilité d'erreur \textbackslash mathbb\{P\}(U \textgreater{} \textbackslash alpha) (solution)}}\label{a-probabilituxe9-derreur-mathbbpu-alpha-solution}}

\begin{itemize}
\tightlist
\item
  \textbf{Avec \(\operatorname{Q}\) -} Le résultat est immédiat puisque
  c'est la définition: \begin{align}
    P(U>\alpha) = \frac{1}{\sqrt{2 \pi}} \int_{\alpha}^{+\infty} e^{\frac{-t^{2}}{2}} \mathrm{d} t
                = \operatorname{Q}(\alpha)
  \end{align}
\end{itemize}

%% \begin{center}\rule{0.5\linewidth}{\linethickness}\end{center}

    \hypertarget{a-probabilituxe9-derreur-mathbbpu-alpha-solution}{%
\subsubsection{\texorpdfstring{(a) Probabilité d'erreur
\(\mathbb{P}(U > \alpha)\)
(solution)}{(a) Probabilité d'erreur \textbackslash mathbb\{P\}(U \textgreater{} \textbackslash alpha) (solution)}}\label{a-probabilituxe9-derreur-mathbbpu-alpha-solution}}

\begin{itemize}
\tightlist
\item
  \textbf{Avec \(\operatorname{erfc}\) -} On va faire le changement de
  variable suivant: \begin{align}
    v = \frac{t}{\sqrt{2}} \Rightarrow v^2 = \frac{t^2}{2}\quad\text{et}\quad \mathrm{d}v = \frac{\mathrm{d}t}{\sqrt{2}}
  \end{align} On obtient alors: \begin{align}
    P(U>\alpha) = \frac{1}{\sqrt{2 \pi}} \int_{\alpha}^{+\infty} e^{\frac{-t^{2}}{2}} \mathrm{d} t
                &= \frac{1}{\sqrt{2 \pi}} \int_{\frac{\alpha}{\sqrt{2}}}^{+\infty} e^{-v^2} \sqrt{2}\mathrm{d} v\\
                &= \frac{1}{\sqrt{\pi}} \int_{\frac{\alpha}{\sqrt{2}}}^{+\infty} e^{-v^2} \mathrm{d} v\\
                &= \frac{1}{2}\times\frac{2}{\sqrt{\pi}} \int_{\frac{\alpha}{\sqrt{2}}}^{+\infty} e^{-v^2} \mathrm{d} v\\
    P(U>\alpha) &= \frac{1}{2}\operatorname{erfc}\left(\frac{\alpha}{\sqrt{2}}\right)
  \end{align}
\end{itemize}

    \hypertarget{b-probabilituxe9-derreur-mathbbpx-alpha}{%
\subsubsection{\texorpdfstring{(b) Probabilité d'erreur
\(\mathbb{P}(X > \alpha)\)}{(b) Probabilité d'erreur \textbackslash mathbb\{P\}(X \textgreater{} \textbackslash alpha)}}\label{b-probabilituxe9-derreur-mathbbpx-alpha}}

Cette fois-ci, on considère \(X\sim\mathcal{N}(0,\sigma_X^2)\).

Les fonctions \(\mathrm{Q}\) et \(\mathrm{erfc}\) sont les suivantes:
\begin{align}
    \operatorname{Q}(x) = \frac{1}{\sqrt{2 \pi}} \int_{x}^{+\infty} e^{\frac{-t^{2}}{2}} \mathrm{d} t \qquad\text{et}\qquad
    \operatorname{erfc}(x)=\frac{2}{\sqrt{\pi}} \int_{x}^{+\infty} e^{-t^{2}} \mathrm{d} t
\end{align}

On peut commencer comme avec \(U\), ou \textbf{être un peu plus malin}.

Essayons de nous ramener au cas précédent.

Comment relier les fonctions de répartition de \(X\) et \(U\) ?

\begin{align}
    \mathbb{P}(X > \alpha) = \mathbb{P}(U > \beta(\alpha))
\end{align}

    \hypertarget{b-probabilituxe9-derreur-mathbbpu-alpha-solution}{%
\subsubsection{\texorpdfstring{(b) Probabilité d'erreur
\(\mathbb{P}(U > \alpha)\)
(solution)}{(b) Probabilité d'erreur \textbackslash mathbb\{P\}(U \textgreater{} \textbackslash alpha) (solution)}}\label{b-probabilituxe9-derreur-mathbbpu-alpha-solution}}

On posera \(U = \frac{X}{\sigma_X}\) qui a bien une variance de \(1\):
\begin{align}
    \mathbb{V}[U] = \mathbb{V}\left[\frac{X}{\sigma_X}\right] = \frac{\mathbb{V}[X]}{\sigma_X^2} = \frac{\sigma_X^2}{\sigma_X^2} = 1
\end{align} et donc \begin{align}
    \mathbb{P}(X > \alpha) = \mathbb{P}\left(U > \frac{\alpha}{\sigma_X}\right)
\end{align}

    \begin{itemize}
\item ~
  \hypertarget{avec-operatornameq---mathbbpleftu-fracalphasigma_xright-operatornameqleftfracalphasigma_xright.}{%
  \subsection{\texorpdfstring{\textbf{Avec \(\operatorname{Q}\) -}
  \(\mathbb{P}\left(U > \frac{\alpha}{\sigma_X}\right) = \operatorname{Q}\left(\frac{\alpha}{\sigma_X}\right)\).}{Avec \textbackslash operatorname\{Q\} - \textbackslash mathbb\{P\}\textbackslash left(U \textgreater{} \textbackslash frac\{\textbackslash alpha\}\{\textbackslash sigma\_X\}\textbackslash right) = \textbackslash operatorname\{Q\}\textbackslash left(\textbackslash frac\{\textbackslash alpha\}\{\textbackslash sigma\_X\}\textbackslash right).}}\label{avec-operatornameq---mathbbpleftu-fracalphasigma_xright-operatornameqleftfracalphasigma_xright.}}
\item
  \textbf{Avec \(\operatorname{erfc}\) -}
  \(\mathbb{P}\left(U > \frac{\alpha}{\sigma_X}\right) = \frac{1}{2}\operatorname{erfc}\left(\frac{\alpha}{\sqrt{2}\sigma_X}\right)\).
\end{itemize}

    \hypertarget{application}{%
\section{3. Application}\label{application}}

On considère une transmission de type BPSK:

\begin{longtable}[]{@{}cc@{}}
\toprule
Bit \(-B\) & Amplitude émise \(-E\)\tabularnewline
\midrule
\endhead
\(1\) & \(2 V\)\tabularnewline
\(0\) & \(0 V\)\tabularnewline
\bottomrule
\end{longtable}

Le canal de communication est bruité.

Il \textbf{ajoute} au signal transmis \(E\) une VAR
\(N\sim\mathcal{N}(0,\sigma^2)\).

On reçoit donc la variable aléatoire \(Y\) définit par: \begin{align}
    Y = E + N
\end{align}

    \hypertarget{a-loi-de-y}{%
\subsubsection{\texorpdfstring{(a) Loi de
\(Y\)}{(a) Loi de Y}}\label{a-loi-de-y}}

Rappelons que

\[\mathbb{E}[Y] = \mathbb{E}[E + N] = \mathbb{E}[E] + \mathbb{E}[N] = E + \mathbb{E}[N]\]

et (\(E\) et \(N\) étant indépendents)

\[\mathbb{V}[Y] = \mathbb{V}[E + N] = \mathbb{V}[E] + \mathbb{V}[N] = \mathbb{V}[N]\]

Quelle est la loi de \(Y\) selon le bit émis ?

    % \begin{center}\rule{0.5\linewidth}{\linethickness}\end{center}

\textbf{Réponse:}

\begin{longtable}[]{@{}ccc@{}}
\toprule
\(B\) & \(E\) & \(Y\)\tabularnewline
\midrule
\endhead
\(1\) & \(2 V\) & \(\mathcal{N}(2,\sigma^2)\)\tabularnewline
\(0\) & \(0 V\) & \(\mathcal{N}(0,\sigma^2)\)\tabularnewline
\bottomrule
\end{longtable}

    \begin{tcolorbox}[breakable, size=fbox, boxrule=1pt, pad at break*=1mm,colback=cellbackground, colframe=cellborder]
\prompt{In}{incolor}{3}{\boxspacing}
\begin{Verbatim}[commandchars=\\\{\}]
\PY{k+kn}{from} \PY{n+nn}{ipywidgets} \PY{k+kn}{import} \PY{o}{*}
\PY{k+kn}{import} \PY{n+nn}{ipywidgets} \PY{k}{as} \PY{n+nn}{widgets}

\PY{k}{def} \PY{n+nf}{pdf\PYZus{}plot}\PY{p}{(}\PY{n}{mu\PYZus{}0}\PY{p}{,} \PY{n}{mu\PYZus{}1}\PY{p}{)}\PY{p}{:}
    \PY{c+c1}{\PYZsh{} Calcules les densités de probabilité pour chaque variable}
    \PY{n}{x} \PY{o}{=} \PY{n}{np}\PY{o}{.}\PY{n}{linspace}\PY{p}{(}\PY{o}{\PYZhy{}}\PY{l+m+mf}{3.0}\PY{p}{,} \PY{l+m+mf}{6.0}\PY{p}{,} \PY{n}{num}\PY{o}{=}\PY{l+m+mi}{1000}\PY{p}{)}\PY{p}{;}
    \PY{n}{y\PYZus{}0} \PY{o}{=} \PY{n}{normpdf}\PY{p}{(}\PY{n}{x}\PY{p}{,} \PY{n}{mu\PYZus{}0}\PY{p}{,} \PY{l+m+mi}{1}\PY{p}{)}\PY{p}{;} \PY{c+c1}{\PYZsh{} densité de probabilité de X}
    \PY{n}{y\PYZus{}1} \PY{o}{=} \PY{n}{normpdf}\PY{p}{(}\PY{n}{x}\PY{p}{,} \PY{n}{mu\PYZus{}1}\PY{p}{,} \PY{l+m+mi}{1}\PY{p}{)}\PY{p}{;} \PY{c+c1}{\PYZsh{} densité de probabilité de X + 2}
    
    \PY{c+c1}{\PYZsh{} Trace les courbes}
    \PY{n}{fig} \PY{o}{=} \PY{n}{plt}\PY{o}{.}\PY{n}{figure}\PY{p}{(}\PY{n}{figsize}\PY{o}{=}\PY{p}{(}\PY{l+m+mi}{12}\PY{p}{,}\PY{l+m+mi}{6}\PY{p}{)}\PY{p}{,} \PY{n}{dpi}\PY{o}{=}\PY{l+m+mi}{100}\PY{p}{)}\PY{p}{;}
    \PY{n}{ax}\PY{o}{=}\PY{n}{plt}\PY{o}{.}\PY{n}{axes}\PY{p}{(}\PY{p}{)}\PY{p}{;}
    \PY{n}{ax}\PY{o}{.}\PY{n}{grid}\PY{p}{(}\PY{p}{)}\PY{p}{;}
    \PY{n}{ax}\PY{o}{.}\PY{n}{plot}\PY{p}{(}\PY{n}{x}\PY{p}{,} \PY{n}{y\PYZus{}0}\PY{p}{,} \PY{n}{label}\PY{o}{=}\PY{l+s+s2}{\PYZdq{}}\PY{l+s+s2}{\PYZdl{}p\PYZus{}}\PY{l+s+s2}{\PYZob{}}\PY{l+s+s2}{Y|B\PYZcb{}(t|0)\PYZdl{}}\PY{l+s+s2}{\PYZdq{}}\PY{p}{)}\PY{p}{;}
    \PY{n}{ax}\PY{o}{.}\PY{n}{plot}\PY{p}{(}\PY{n}{x}\PY{p}{,} \PY{n}{y\PYZus{}1}\PY{p}{,} \PY{n}{label}\PY{o}{=}\PY{l+s+s2}{\PYZdq{}}\PY{l+s+s2}{\PYZdl{}p\PYZus{}}\PY{l+s+s2}{\PYZob{}}\PY{l+s+s2}{Y|B\PYZcb{}(t|1)\PYZdl{}}\PY{l+s+s2}{\PYZdq{}}\PY{p}{)}\PY{p}{;}
    
    \PY{c+c1}{\PYZsh{} Trace les zones d\PYZsq{}erreurs}
    \PY{n}{ax}\PY{o}{.}\PY{n}{fill\PYZus{}between}\PY{p}{(}\PY{n}{x}\PY{p}{[}\PY{n}{y\PYZus{}0} \PY{o}{\PYZgt{}}\PY{o}{=} \PY{n}{y\PYZus{}1}\PY{p}{]}\PY{p}{,} \PY{n}{y\PYZus{}1}\PY{p}{[}\PY{n}{y\PYZus{}0} \PY{o}{\PYZgt{}}\PY{o}{=} \PY{n}{y\PYZus{}1}\PY{p}{]}\PY{p}{,} \PY{n}{facecolor}\PY{o}{=}\PY{l+s+s2}{\PYZdq{}}\PY{l+s+s2}{orange}\PY{l+s+s2}{\PYZdq{}}\PY{p}{,} \PY{n}{alpha}\PY{o}{=}\PY{o}{.}\PY{l+m+mi}{5}\PY{p}{,} \PY{n}{label}\PY{o}{=}\PY{l+s+s2}{\PYZdq{}}\PY{l+s+s2}{\PYZdl{}}\PY{l+s+s2}{\PYZbs{}}\PY{l+s+s2}{hat}\PY{l+s+si}{\PYZob{}B\PYZcb{}}\PY{l+s+s2}{=1 }\PY{l+s+s2}{\PYZbs{}}\PY{l+s+s2}{mid  B=0\PYZdl{}}\PY{l+s+s2}{\PYZdq{}}\PY{p}{)}
    \PY{n}{ax}\PY{o}{.}\PY{n}{fill\PYZus{}between}\PY{p}{(}\PY{n}{x}\PY{p}{[}\PY{n}{y\PYZus{}0} \PY{o}{\PYZlt{}} \PY{n}{y\PYZus{}1}\PY{p}{]}\PY{p}{,} \PY{n}{y\PYZus{}0}\PY{p}{[}\PY{n}{y\PYZus{}0} \PY{o}{\PYZlt{}} \PY{n}{y\PYZus{}1}\PY{p}{]}\PY{p}{,} \PY{n}{facecolor}\PY{o}{=}\PY{l+s+s2}{\PYZdq{}}\PY{l+s+s2}{blue}\PY{l+s+s2}{\PYZdq{}}\PY{p}{,} \PY{n}{alpha}\PY{o}{=}\PY{o}{.}\PY{l+m+mi}{5}\PY{p}{,} \PY{n}{label}\PY{o}{=}\PY{l+s+s2}{\PYZdq{}}\PY{l+s+s2}{\PYZdl{}}\PY{l+s+s2}{\PYZbs{}}\PY{l+s+s2}{hat}\PY{l+s+si}{\PYZob{}B\PYZcb{}}\PY{l+s+s2}{=0 }\PY{l+s+s2}{\PYZbs{}}\PY{l+s+s2}{mid  B=1\PYZdl{}}\PY{l+s+s2}{\PYZdq{}}\PY{p}{)}
    
    \PY{c+c1}{\PYZsh{} Paramètrage de la figure}
    \PY{n}{ax}\PY{o}{.}\PY{n}{set\PYZus{}xlabel}\PY{p}{(}\PY{l+s+s2}{\PYZdq{}}\PY{l+s+s2}{\PYZdl{}y\PYZdl{}}\PY{l+s+s2}{\PYZdq{}}\PY{p}{)}\PY{p}{;}
    \PY{n}{ax}\PY{o}{.}\PY{n}{set\PYZus{}ylabel}\PY{p}{(}\PY{l+s+s2}{\PYZdq{}}\PY{l+s+s2}{\PYZdl{}p\PYZus{}}\PY{l+s+s2}{\PYZob{}}\PY{l+s+s2}{Y|B\PYZcb{}(t}\PY{l+s+s2}{\PYZbs{}}\PY{l+s+s2}{mid b)\PYZdl{}}\PY{l+s+s2}{\PYZdq{}}\PY{p}{)}\PY{p}{;}
    \PY{n}{ax}\PY{o}{.}\PY{n}{legend}\PY{p}{(}\PY{p}{)}\PY{p}{;}
    \PY{n}{plt}\PY{o}{.}\PY{n}{show}\PY{p}{(}\PY{p}{)}\PY{p}{;}
\end{Verbatim}
\end{tcolorbox}

    \hypertarget{b-probabilituxe9-derreur}{%
\subsubsection{(b) Probabilité
d'erreur}\label{b-probabilituxe9-derreur}}

    \begin{tcolorbox}[breakable, size=fbox, boxrule=1pt, pad at break*=1mm,colback=cellbackground, colframe=cellborder]
\prompt{In}{incolor}{4}{\boxspacing}
\begin{Verbatim}[commandchars=\\\{\}]
\PY{n}{interact}\PY{p}{(}\PY{n}{pdf\PYZus{}plot}\PY{p}{,}
    \PY{n}{mu\PYZus{}0}\PY{o}{=}\PY{n}{widgets}\PY{o}{.}\PY{n}{FloatSlider}\PY{p}{(}\PY{n}{value}\PY{o}{=}\PY{l+m+mi}{0}\PY{p}{,}\PY{n+nb}{min}\PY{o}{=}\PY{o}{\PYZhy{}}\PY{l+m+mf}{1.0}\PY{p}{,}\PY{n+nb}{max}\PY{o}{=}\PY{l+m+mf}{1.0}\PY{p}{,}\PY{n}{step}\PY{o}{=}\PY{l+m+mf}{0.1}\PY{p}{)}\PY{p}{,}
    \PY{n}{mu\PYZus{}1}\PY{o}{=}\PY{n}{widgets}\PY{o}{.}\PY{n}{FloatSlider}\PY{p}{(}\PY{n}{value}\PY{o}{=}\PY{l+m+mi}{2}\PY{p}{,}\PY{n+nb}{min}\PY{o}{=}\PY{l+m+mf}{1.0}\PY{p}{,}\PY{n+nb}{max}\PY{o}{=}\PY{l+m+mf}{3.0}\PY{p}{,}\PY{n}{step}\PY{o}{=}\PY{l+m+mf}{0.1}\PY{p}{)}\PY{p}{,}
\PY{p}{)}\PY{p}{;}
\end{Verbatim}
\end{tcolorbox}

    
    \begin{verbatim}
interactive(children=(FloatSlider(value=0.0, description='mu_0', max=1.0, min=-1.0), FloatSlider(value=2.0, de…
    \end{verbatim}

    
    \hypertarget{c-valeurs-de-bruit-causant-une-erreur}{%
\subsubsection{(c) Valeurs de bruit causant une
erreur}\label{c-valeurs-de-bruit-causant-une-erreur}}

D'après la figure précédente, on voit qu'il y a une erreur lorsque: - Si
\(B=0\): \(e_0 + N > \lambda \Rightarrow N > \lambda - e_0\) - Si
\(B=1\): \(e_1 + N < \lambda \Rightarrow N < \lambda - e_1\)

    \hypertarget{d-probabilituxe9-derreur-binaire}{%
\subsubsection{(d) Probabilité d'erreur
binaire}\label{d-probabilituxe9-derreur-binaire}}

Pour calculer la probabilité d'erreur binaire, on effectuera les étapes
suivantes:

\begin{itemize}
\item
  \textbf{Proba. totales et Bayes}
\item
  \textbf{Faire apparaître \(N\)}
\item
  \textbf{Utiliser les propriétés de la gaussienne}
\end{itemize}

    \hypertarget{d-probabilituxe9-derreur-binaire}{%
\subsubsection{(d) Probabilité d'erreur
binaire}\label{d-probabilituxe9-derreur-binaire}}

Pour calculer la probabilité d'erreur binaire, on effectuera les étapes
suivantes:

\begin{itemize}
\item
  \textbf{Proba. totales et Bayes} \begin{align}
    P_e &= \mathbb{P}(B=0 \cap Y>\lambda)+\mathbb{P}(B=1 \cap Y<\lambda)&\text{(formule des probabilités totales})\\
        &= \mathbb{P}(Y>\lambda \mid B=0) \mathbb{P}(B=0)+\mathbb{P}(Y<\lambda \mid B=1) \mathrm{P}(B=1) &\text{(formule de Bayes})\\
        &= \mathbb{P}(Y>\lambda \mid E=e_0) \mathbb{P}(B=0)+\mathbb{P}(Y<\lambda \mid E=e_1) \mathrm{P}(B=1)\\
        &= \frac{1}{2}\left(\mathbb{P}\left(Y>\lambda \mid E=e_0\right) + \mathbb{P}\left(Y<\lambda \mid E=e_1\right)\right)
  \end{align}
\item
  \emph{Faire apparaître \(N\)}
\item
  \emph{Utiliser les propriétés de la gaussienne}
\end{itemize}

    \hypertarget{d-probabilituxe9-derreur-binaire}{%
\subsubsection{(d) Probabilité d'erreur
binaire}\label{d-probabilituxe9-derreur-binaire}}

Pour calculer la probabilité d'erreur binaire, on effectuera les étapes
suivantes:

\begin{itemize}
\item
  \emph{Proba. totales et Bayes} \begin{align}
    P_e &= \frac{1}{2}\left(\mathbb{P}\left(Y>\lambda \mid E=e_0\right) + \mathbb{P}\left(Y<\lambda \mid E=e_1\right)\right)
  \end{align}
\item
  \textbf{Faire apparaître \(N\)} \begin{align}
     P_e &= \frac{1}{2}\left(\mathbb{P}\left(e_0 + N>\lambda \mid E=e_0\right) + \mathbb{P}\left(e_1 + N<\lambda \mid E=e_1\right)\right)\\
        &= \frac{1}{2}\left(\mathbb{P}\left(N>\lambda-e_0 \mid E=e_0\right) + \mathbb{P}\left(N<\lambda-e_1 \mid E=e_1\right)\right)\\
        &= \frac{1}{2}\left(\mathbb{P}\left(N>\frac{e_1-e_0}{2} \mid E=e_0\right) + \mathbb{P}\left(N<\frac{e_0-e_1}{2} \mid E=e_1\right)\right)
  \end{align}
\item
  \emph{Utiliser les propriétés de la gaussienne}
\end{itemize}

    \hypertarget{d-probabilituxe9-derreur-binaire}{%
\subsubsection{(d) Probabilité d'erreur
binaire}\label{d-probabilituxe9-derreur-binaire}}

Pour calculer la probabilité d'erreur binaire, on effectuera les étapes
suivantes:

\begin{itemize}
\item
  \emph{Proba. totales et Bayes} \begin{align}
    P_e &= \mathbb{P}(Y>\lambda \mid E=e_0) \mathbb{P}(B=0)+\mathbb{P}(Y<\lambda \mid E=e_1) \mathrm{P}(B=1)
  \end{align}
\item
  \emph{Faire apparaître \(N\)} \begin{align}
     P_e &= \frac{1}{2}\left(\mathbb{P}\left(N>\frac{e_1-e_0}{2} \mid E=e_0\right) + \mathbb{P}\left(N<\frac{e_0-e_1}{2} \mid E=e_1\right)\right)
  \end{align}
\item
  \textbf{Utiliser les propriétés de la gaussienne} \begin{align}
    P_e &= \frac{1}{2}\left(\mathbb{P}\left(N>\frac{e_1-e_0}{2} \mid E=e_0\right) + \mathbb{P}\left(N<-\frac{e_1-e_0}{2} \mid E=e_1\right)\right)\\
        &= \frac{1}{2}\left(\mathbb{P}\left(N>\frac{e_1-e_0}{2} \mid E=e_0\right) + \mathbb{P}\left(N>\frac{e_1-e_0}{2} \mid E=e_1\right)\right) &\text{(la gaussienne est symétrique)}\\
        &= \mathbb{P}\left(N>\frac{e_1-e_0}{2} \mid E=e_0\right)\\
        &= \operatorname{Q}\left(\frac{e_1-e_0}{2\sigma}\right)&\text{(d'après la question 2.(b))}
  \end{align}
\end{itemize}

    \hypertarget{d-probabilituxe9-derreur-binaire}{%
\subsubsection{(d) Probabilité d'erreur
binaire}\label{d-probabilituxe9-derreur-binaire}}

Pour calculer la probabilité d'erreur binaire, on effectuera les étapes
suivantes:

\begin{itemize}
\item
  \emph{Proba. totales et Bayes} \begin{align}
    P_e &= \mathbb{P}(Y>\lambda \mid E=e_0) \mathbb{P}(B=0)+\mathbb{P}(Y<\lambda \mid E=e_1) \mathrm{P}(B=1)
  \end{align}
\item
  \emph{Faire apparaître \(N\)} \begin{align}
     P_e &= \frac{1}{2}\left(\mathbb{P}\left(N>\frac{e_1-e_0}{2} \mid E=e_0\right) + \mathbb{P}\left(N<\frac{e_0-e_1}{2} \mid E=e_1\right)\right)
  \end{align}
\item
  \emph{Utiliser les propriétés de la gaussienne} \begin{align}
    P_e &= \operatorname{Q}\left(\frac{e_1-e_0}{2\sigma}\right)
  \end{align}
\end{itemize}

\hypertarget{on-obtient-un-ruxe9sultat-simple-qui-duxe9pend-directement-de-luxe9cart-e_1---e_0}{%
\paragraph{\texorpdfstring{On obtient un résultat simple qui dépend
directement de l'écart \(e_1 - e_0\)
!}{On obtient un résultat simple qui dépend directement de l'écart e\_1 - e\_0 !}}\label{on-obtient-un-ruxe9sultat-simple-qui-duxe9pend-directement-de-luxe9cart-e_1---e_0}}

    \hypertarget{e-bit-error-rate-ber-et-snr}{%
\subsubsection{(e) Bit Error Rate (BER) et
SNR}\label{e-bit-error-rate-ber-et-snr}}

On admet que la puissance du signal et celle du bruit sont égales à
\begin{align}
    P_S = \frac{e_0^2 + e_1^2}{2}\quad\text{et}\quad P_N = \sigma^2
\end{align} Le SNR vaut donc \begin{align}
    \mathsf{SNR} = \frac{P_S}{\sigma^2}
\end{align}

\begin{itemize}
\tightlist
\item
  \textbf{Si \(e_0 = 0\):} \(P_S = \frac{e_1^2}{2}\) et
  \(\mathsf{SNR} = \frac{e_1^2}{2\sigma^2} \Rightarrow P_e = \mathrm{Q}\left(\sqrt{\frac{\mathrm{SNR}}{2}}\right)\)
\item
  \textbf{Si \(e_0 = -e_1\):} \(P_S = e_1^2\) et
  \(\mathsf{SNR} = \frac{e_1^2}{\sigma^2} \Rightarrow P_e = \mathrm{Q}\left(\sqrt{\mathrm{SNR}}\right)\)
\end{itemize}

    \hypertarget{bonus---tracuxe9-de-p_e}{%
\section{\texorpdfstring{Bonus - tracé de
\(P_e\)}{Bonus - tracé de P\_e}}\label{bonus---tracuxe9-de-p_e}}

    \begin{tcolorbox}[breakable, size=fbox, boxrule=1pt, pad at break*=1mm,colback=cellbackground, colframe=cellborder]
\prompt{In}{incolor}{5}{\boxspacing}
\begin{Verbatim}[commandchars=\\\{\}]
\PY{k+kn}{from} \PY{n+nn}{scipy}\PY{n+nn}{.}\PY{n+nn}{stats} \PY{k+kn}{import} \PY{n}{norm}
\PY{k}{def} \PY{n+nf}{plot\PYZus{}BER}\PY{p}{(}\PY{n}{snr}\PY{p}{)}\PY{p}{:}
    \PY{c+c1}{\PYZsh{} Generate a normal random variable}
    \PY{n}{rv} \PY{o}{=} \PY{n}{norm}\PY{p}{(}\PY{p}{)}
    
    \PY{c+c1}{\PYZsh{} Trace les courbes}
    \PY{n}{fig} \PY{o}{=} \PY{n}{plt}\PY{o}{.}\PY{n}{figure}\PY{p}{(}\PY{n}{figsize}\PY{o}{=}\PY{p}{(}\PY{l+m+mi}{16}\PY{p}{,}\PY{l+m+mi}{6}\PY{p}{)}\PY{p}{,} \PY{n}{dpi}\PY{o}{=}\PY{l+m+mi}{100}\PY{p}{)}\PY{p}{;}
    \PY{n}{ax}\PY{o}{=}\PY{n}{plt}\PY{o}{.}\PY{n}{axes}\PY{p}{(}\PY{p}{)}\PY{p}{;}
    \PY{n}{ax}\PY{o}{.}\PY{n}{grid}\PY{p}{(}\PY{p}{)}\PY{p}{;}
    \PY{n}{ax}\PY{o}{.}\PY{n}{semilogy}\PY{p}{(}\PY{n}{snr}\PY{p}{,} \PY{n}{rv}\PY{o}{.}\PY{n}{pdf}\PY{p}{(}\PY{n}{np}\PY{o}{.}\PY{n}{sqrt}\PY{p}{(}\PY{n}{snr}\PY{o}{/}\PY{l+m+mi}{2}\PY{p}{)}\PY{p}{)}\PY{p}{,} \PY{l+s+s1}{\PYZsq{}}\PY{l+s+s1}{k\PYZhy{}\PYZhy{}}\PY{l+s+s1}{\PYZsq{}}\PY{p}{,} \PY{n}{label}\PY{o}{=}\PY{l+s+s2}{\PYZdq{}}\PY{l+s+s2}{\PYZdl{}}\PY{l+s+s2}{\PYZbs{}}\PY{l+s+s2}{mathrm}\PY{l+s+si}{\PYZob{}Q\PYZcb{}}\PY{l+s+s2}{(}\PY{l+s+s2}{\PYZbs{}}\PY{l+s+s2}{sqrt}\PY{l+s+s2}{\PYZob{}}\PY{l+s+s2}{\PYZbs{}}\PY{l+s+s2}{mathsf}\PY{l+s+si}{\PYZob{}SNR\PYZcb{}}\PY{l+s+s2}{/2\PYZcb{})\PYZdl{}}\PY{l+s+s2}{\PYZdq{}}\PY{p}{)}
    \PY{n}{ax}\PY{o}{.}\PY{n}{semilogy}\PY{p}{(}\PY{n}{snr}\PY{p}{,} \PY{n}{rv}\PY{o}{.}\PY{n}{pdf}\PY{p}{(}\PY{n}{np}\PY{o}{.}\PY{n}{sqrt}\PY{p}{(}\PY{n}{snr}\PY{p}{)}\PY{p}{)}\PY{p}{,} \PY{l+s+s1}{\PYZsq{}}\PY{l+s+s1}{k\PYZhy{}}\PY{l+s+s1}{\PYZsq{}}\PY{p}{,} \PY{n}{label}\PY{o}{=}\PY{l+s+s2}{\PYZdq{}}\PY{l+s+s2}{\PYZdl{}}\PY{l+s+s2}{\PYZbs{}}\PY{l+s+s2}{mathrm}\PY{l+s+si}{\PYZob{}Q\PYZcb{}}\PY{l+s+s2}{(}\PY{l+s+s2}{\PYZbs{}}\PY{l+s+s2}{sqrt}\PY{l+s+s2}{\PYZob{}}\PY{l+s+s2}{\PYZbs{}}\PY{l+s+s2}{mathsf}\PY{l+s+si}{\PYZob{}SNR\PYZcb{}}\PY{l+s+s2}{\PYZcb{})\PYZdl{}}\PY{l+s+s2}{\PYZdq{}}\PY{p}{)}
    
    \PY{c+c1}{\PYZsh{} Paramètrage de la figure}
    \PY{n}{ax}\PY{o}{.}\PY{n}{set\PYZus{}xlabel}\PY{p}{(}\PY{l+s+s2}{\PYZdq{}}\PY{l+s+s2}{\PYZdl{}}\PY{l+s+s2}{\PYZbs{}}\PY{l+s+s2}{mathsf}\PY{l+s+si}{\PYZob{}SNR\PYZcb{}}\PY{l+s+s2}{\PYZdl{}}\PY{l+s+s2}{\PYZdq{}}\PY{p}{)}\PY{p}{;}
    \PY{n}{ax}\PY{o}{.}\PY{n}{set\PYZus{}ylabel}\PY{p}{(}\PY{l+s+s2}{\PYZdq{}}\PY{l+s+s2}{\PYZdl{}}\PY{l+s+s2}{\PYZbs{}}\PY{l+s+s2}{mathrm}\PY{l+s+si}{\PYZob{}Q\PYZcb{}}\PY{l+s+s2}{\PYZdl{}}\PY{l+s+s2}{\PYZdq{}}\PY{p}{)}\PY{p}{;}
    \PY{n}{ax}\PY{o}{.}\PY{n}{legend}\PY{p}{(}\PY{p}{)}\PY{p}{;}
    \PY{n}{plt}\PY{o}{.}\PY{n}{show}\PY{p}{(}\PY{p}{)}\PY{p}{;}
\end{Verbatim}
\end{tcolorbox}

    \begin{tcolorbox}[breakable, size=fbox, boxrule=1pt, pad at break*=1mm,colback=cellbackground, colframe=cellborder]
\prompt{In}{incolor}{6}{\boxspacing}
\begin{Verbatim}[commandchars=\\\{\}]
\PY{n}{plot\PYZus{}BER}\PY{p}{(}\PY{n}{np}\PY{o}{.}\PY{n}{linspace}\PY{p}{(}\PY{l+m+mi}{0}\PY{p}{,}\PY{l+m+mi}{20}\PY{p}{,}\PY{l+m+mi}{1000}\PY{p}{)}\PY{p}{)}
\end{Verbatim}
\end{tcolorbox}

    \begin{center}
    \adjustimage{max size={0.9\linewidth}{0.9\paperheight}}{output_34_0.png}
    \end{center}
    { \hspace*{\fill} \\}
    

    % Add a bibliography block to the postdoc
    
    
    
\end{document}
